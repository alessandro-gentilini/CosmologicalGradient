% This file is part of the CosmologicalGradient project.
% Copyright 2018 the authors.

\documentclass[12pt, letterpaper]{article}

% margins
\addtolength{\topmargin}{-0.75in}
\addtolength{\textheight}{1.50in}

% language
\newcommand{\foreign}[1]{\textsl{#1}}
\newcommand{\etc}{\foreign{etc}}

% math
\newcommand{\inv}{^{-1}}
\newcommand{\T}{^{\mathsf{T}}}

\begin{document}

\paragraph{abstract:}
% Context:
Cosmological homogeneity is well established in galaxy surveys in a mean-density
sense.
At the same time, there are suggestions of small anisotropies in clustering power in
the microwave background data, and inflation-like theories might naturally produce
gradients in physical parameters on extremely large scales.
% Aims:
Here we ask about the statistical homogeneity of the clustering properties of galaxies.
That is: Is there a large-scale gradient in the correlation function of galaxies
apparent in the well-observed part of the Hubble volume?
% Method:
We generate an estimator (implicitly part of a large family of estimators)
that delivers simultaneously the galaxy--galaxy auto-correlation function
(or indeed any cross-corelation function)
and it's spatial gradient.
The estimator is general; we apply it to the XXX sample of the YYY survey.
% Results:
We don't find any significant gradient at the ZZZ level.
The implications of the result are discussed in the context of contemporary ideas
about physical cosmology and inflation.

\section{Introduction}

What do we know about galaxy homogeneity?

What are the suggestions in the CMB or other places for anisotropies?

What are the theoretical ideas about super-horizon gradients in physical laws?

\section{Methods}

Recall:
\begin{eqnarray}\displaystyle
\xi_k &\leftarrow& \frac{DD_k - 2\,DR_k + RR_k}{RR_k}
\\
DD_k &\equiv& \sum_{n n'} i(g_k < |x_n - x_{n'}| < h_k)
\\
DR_k &\equiv& \sum_{n m} i(g_k < |x_n - x_m| < h_k)
\\
RR_k &\equiv& \sum_{m m'} i(g_k < |x_m - x_{m'}| < h_k)
\quad ,
\end{eqnarray}
where
$\xi_k$ is the value of the correlation function estimate in radius bin $k$,
$DD_k$ is the count of data--data pairs in bin $k$
(which has bin edges $g_k$ and $h_k$),
$i()$ is an indicator function,
$x_n$, $x_n'$, \etc, are three-d positions (or two-d projected, depending),
$DR_k$ is the count of data--random pairs,
and
$RR_k$ is the count of random--random pairs.

Now we are going to estimate a vector of quantities.
Our model is that the correlation function has a linear gradient in it, or
\begin{eqnarray}\displaystyle
\xi_k(s) &=& a_k\T\cdot u
\\
u &\equiv& \left[\begin{array}{c}1 \\ (s - s_0)\end{array}\right]
\end{eqnarray}
where
again $\xi_k(s)$ is the correlation function in radius bin $k$,
but now it depends on redshift-space three-vector position $s$,
$a_k$ is a four-vector of amplitudes relevant to bin $k$,
$u$ is a four-vector ``design matrix'',
$s$ is the three-vector redshift-space three-vector position
at which the correlation function is being evaluated,
and
$s_0$ is a fiducial redshift-space three-vector position (the ``pivot'', perhaps).
(Implicitly all vectors are column vectors in my world.)
To be clear, the first element of $a_k$ is the fiducial value of the correlation
function---the value the function has at fiducial position $s_0$---and the
following three elements of $a_k$ are the gradient of $\xi_k$ with respect to
redshift-space three-vector position $s$.

The estimator for the four-vector $a_k$ becomes
\begin{eqnarray}\displaystyle
a_k &\leftarrow& Q_k\inv\cdot [DD_k - 2\,DR_k + RR_k]
\\
DD_k &\equiv& \sum_{n n'} i(g_k < |x_n - x_{n'}| < h_k)\,u_{n n'}
\\
DR_k &\equiv& \sum_{n m} i(g_k < |x_n - x_m| < h_k)\,u_{n m}
\\
RR_k &\equiv& \sum_{m m'} i(g_k < |x_m - x_{m'}| < h_k)\,u_{m m'}
\\
Q_k &\equiv& \sum_{m m'} i(g_k < |x_m - x_{m'}| < h_k)\,u_{m m'} \cdot u_{m m'}\T
\quad ,
\end{eqnarray}
where
everything is the same, except:
Now the indicator functions are multiplied by four-vectors $u_{n n'}$ which are
four-vector design vectors appropriate for each pair (constructed at the center-point
of each pair, say),
and
instead of dividing by $RR_k$ we left-multiply by an inverse $4\times 4$ matrix
constructed by outer products of $u_{m m'}$ four-vectors.
(Because vectors are column vectors for me, $u\cdot u\T$ is an outer product;
$u\T\cdot u$ would be an inner product.)

\section{Data and results}

\section{Discussion}

\end{document}
